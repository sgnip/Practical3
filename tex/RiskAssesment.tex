% -*- root: ../RiskAnalysis.tex -*-

\subsection{Justification of risk priorities}

The risks identified in section \ref{secRiskIdentification} are presented below, with the priority has been calculated using the product of the probability and the risk impact. The higher priority risks are highlighted.

\begin{table}[hbtp]
\centering
\begin{tabular}{|c|c|c|c|}
\hline
\textbf{Risk} & \textbf{Probability} & \textbf{Impact} & \textbf{Priority} \\ \hline \hline
\riskcalc{riskReqChange} \\ \hline
\riskcalc{riskAttitude} \\ \hline
\riskcalc{riskFeaturesUnexpected} \\ \hline
\riskcalc{riskExpectations} \\ \hline
\riskcalc{riskLocalization} \\ \hline % 5
\riskcalc{riskAuthServer} \\ \hline
\riskcalc{riskPhone} \\ \hline
\rowcolor{tred} \riskcalc{riskPrec} \\ \hline
\riskcalc{riskAlgorithms} \\ \hline
\riskcalc{riskRealTime} \\ \hline % 10
\riskcalc{riskUserLoad} \\ \hline
\riskcalc{riskCollaboration} \\ \hline
\riskcalc{riskIntegrationTests} \\ \hline
\rowcolor{tred} \riskcalc{riskAssignment} \\ \hline
\riskcalc{riskManagement} \\ \hline % 15
\rowcolor{tred} \riskcalc{riskDelays} \\ \hline
\riskcalc{riskBudget} \\ \hline
\riskcalc{riskMotivation} \\ \hline
\rowcolor{tred} \riskcalc{riskQuality} \\ \hline
\riskcalc{riskPersonnelRotation} \\ \hline % 20
\rowcolor{tred} \riskcalc{riskPersonnelUnderestimation} \\ \hline
\riskcalc{riskResponsibilitesAssignment} \\ \hline
\end{tabular}
\end{table}

