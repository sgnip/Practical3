% -*- root: ../ProjectPlan.tex -*-
\section{Project Definition}

FML is based on the jolly cooperation between the users of the facilities of the UAM campus and the maintenance staff in charge of them.

As the users will be the ones that will detect the sooner faults on the facilities the campus offers them, they are also the most suitable to report the problems they are having, in order to getting them fixed as soon as possible. With the FML web application, it will only take less than a minute to fill the form and send it to the maintenance staff.

Using the reports of faults detected in the campus, the maintenance personnel will stop losing its precious time revising the installations looking for faults and will be able to focus on the repairs. Apart from this benefit, the maintenance staff will also have a better way to coordinate efforts, as the FML will provide an automatic assignment system to assign repairs to each of the members avoiding overloading any of them and taking into account their distance to the problem, saving time on displacements.

In summary, the users will be able to easily report faults on facilities they are using in order to have them fixed in the less time possible, while the maintenance personnel will multiply its current performance as the majority of their resources will stop being wasted on revisions, but on repairs.

