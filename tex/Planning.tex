% -*- root: ../ProjectPlan.tex -*-
\section{Time estimation and Schedule}
\label{secTimeEstimation}
\subsection{Gantt diagram}

\begin{figure}[hbtp]
\centering
\includegraphics[width=0.8\textwidth]{img/GanttDiagram.png}
\caption{Simplified Gantt diagram.}
\label{figGanttSimple}
\end{figure}

Figure \ref{figGanttSimple} shows a simplified view of the Gantt diagram that represents the schedule of the project. A detailed version of this diagram is included in appendix \ref{chapGantt}.

\subsection{Increment planning}

We have decided to break down the development in 3 increments, each one dedicated to various subsystems. The details of the function points assigned to each increment and the corresponding effort is detailed in table \ref{tblIncrementsSubsystems}. Table \ref{tblSubsystemsAssignedIncrement} reflects the increment in which each subsystem will be completed and the corresponding percentage of each increment's effort dedicated to it.

\begin{table}[hbtp]
\centering
\input{tex/IncrementOutline.tex}
\caption{Detail of the increments and corresponding effort.}
\label{tblIncrementsSubsystems}
\end{table}

\begin{table}[hbtp]
\centering

\begin{tabular}{l|c|c}
\textbf{Subsystem} & \textbf{Increment} & \textbf{Effort \%}  \\ \hline
Task Management & 1 & 100 \% \\
Reporting & 2 & 28 \% \\
Notification and Messaging & 2 & 72 \% \\
User Management & 3 & 36 \% \\
Faults History and Stats & 3 & 64 \% \\
\end{tabular}

\caption{Assigned increment and effort for each subsystem.}
\label{tblSubsystemsAssignedIncrement}
\end{table}

The effort for each phase of each increment is detailed in table \ref{tblIncrementPhases}.

\begin{table}[hbtp]
\centering

\begin{tabular}{|c|c|c|c|}
\hline
\textbf{Increment} & \textbf{Phase} & \textbf{Effort \%} & \textbf{Effort (person-days)} \\ \hline \hline

\multirow{7}{*}{\textsc{Increment 1}} & Analysis & 20 \% & 27.6 \\ \cline{2-4}
& Design & 20 \% & 27.6 \\ \cline{2-4}
& Coding & 20 \% & 27.6 \\ \cline{2-4}
& Unit tests & 10 \% & 13.8 \\ \cline{2-4}
& Integration tests & 20 \% & 27.6 \\ \cline{2-4}
& Implementation & 10 \% & 13.8 \\ \cline{2-4}
& \textit{Total} & \textit{100\%} & \textit{138.6836} \\ \hline \hline

\multirow{7}{*}{\textsc{Increment 2}} & Analysis & 20 \% & 14.20 \\ \cline{2-4}
& Design & 20 \% & 14.20 \\ \cline{2-4}
& Coding & 20 \% & 14.20 \\ \cline{2-4}
& Unit tests & 10 \% & 7.1 \\ \cline{2-4}
& Integration tests & 20 \% & 14.20 \\ \cline{2-4}
& Implementation & 10 \% & 7.1 \\ \cline{2-4}
& \textit{Total} & \textit{100\%} & \textit{70.9544} \\ \hline \hline

\multirow{7}{*}{\textsc{Increment 2}} & Analysis & 20 \% & 10.96 \\ \cline{2-4}
& Design & 20 \% & 10.96 \\ \cline{2-4}
& Coding & 20 \% & 10.96 \\ \cline{2-4}
& Unit tests & 10 \% & 5.4820 \\ \cline{2-4}
& Integration tests & 20 \% & 10.96 \\ \cline{2-4}
& Implementation & 10 \% & 5.4820 \\ \cline{2-4}
& \textit{Total} & \textit{100\%} & \textit{54.8284} \\ \hline

\end{tabular}

\caption{Detail of the increments with the corresponding phases for each one.}
\label{tblIncrementPhases}
\end{table}
