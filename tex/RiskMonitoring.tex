% -*- root: ../RiskAnalysis.tex -*-

As stated in the Project Plan \cite{plan15}, meetings will be hold at the end of each phase of each increment. The reserved dates are shown in table \ref{tblPlannedMeetings}. These meetings, apart from serving as a method to review the deliverables of the corresponding phase, can (and should) also be used to evaluate the current state of risks of the project. If, as stated in section \ref{secBreakpoints}, the meeting corresponds to a breaking point, a decision of whether to cancel the project should be made.

The project manager must be present in meetings corresponding to breakpoints. He should evaluate the current state of the project and the available and applied mitigation measures in order to make the decision. If the cancellation of the project is decided, it should be communicated to Triforce executive and to the client.

\begin{table}[hbtp]
\centering
\begin{tabular}{|c|c|c|c|}
\hline
\textbf{Increment} & \textbf{Activity} & \textbf{Responsible} & \textbf{Date} \\ \hline \hline

\multirow{5}{*}{\textsc{Increment 1}} & Analysis & System Analyst & 12 / 6 / 2015 \\ \cline{2-4}
 & Design & Senior Designer & 24 / 6 / 2015 \\ \cline{2-4}
 & Coding & Senior Programmer & 8 / 7 / 2015 \\ \cline{2-4}
 & Unit Testing & System Analyst & 15 / 7 / 2015 \\ \cline{2-4}
 & Integration Testing & System Analyst & 27 / 7 / 2015 \\ \cline{2-4}
 & Deployment & Systems Tech & 6 / 8 / 2015 \\ \hline \hline

\multirow{5}{*}{\textsc{Increment 2}} & Analysis & System Analyst & 27 / 8 / 2015 \\ \cline{2-4}
 & Design & Senior Designer & 4 / 9 / 2015 \\ \cline{2-4}
 & Coding & Senior Programmer & 14 / 9 / 2015 \\ \cline{2-4}
 & Unit Testing & System Analyst & 17 / 9 / 2015 \\ \cline{2-4}
 & Integration Testing & System Analyst & 24 / 9 / 2015 \\ \cline{2-4}
 & Deployment & Systems Tech & 30 / 9 / 2015 \\ \hline \hline

\multirow{5}{*}{\textsc{Increment 3}} & Analysis & System Analyst & 20 / 10 / 2015 \\ \cline{2-4}
 & Design & Senior Designer & 23 / 10 / 2015 \\ \cline{2-4}
 & Coding & Senior Programmer & 30 / 10 / 2015 \\ \cline{2-4}
 & Unit Testing & System Analyst & 4 / 10 / 2015 \\ \cline{2-4}
 & Integration Testing & System Analyst & 9 / 11 / 2015 \\ \cline{2-4}
 & Deployment & Systems Tech & 13 / 11 / 2015 \\ \hline

\end{tabular}
\caption{Planned project meetings.}
\label{tblPlannedMeetings}
\end{table}

\section{Control and tracking plans}

The main purposes of risk control processes are:
\begin{itemize} 
\item Updating the risk register just as the project development progresses, which means identifying and analyzing new risks that may emerge, elaborating contingency plans for them.
\item Verifying it any of the identified risks has realized; in this case, activate its corresponding response plans.
\item Making the tracking of the currently activated response plans in order to better control their evolution.
\item Administrating the contingency reserve fund (in case it exists). 
\end{itemize}

For risks related with the final system and our ability to develop it in time (\rref{riskPrec}, \rref{riskQuality}, \rref{riskDelays}), emphasis must be placed in the testing of the system. This is the only way to accurately evaluate the actual state of the project, without taking into account subjective opinions that may be prone to errors and misconceptions. As a side effect and as commented in section \ref{secMitigation}, testing will also help to prevent and mitigate the risks.

Apart from testing, the project leader should follow closely the development of the project, so problems are detected as soon as possible and mitigation measures can be taken.

For risks related with the coordination between different teams (\rref{riskAssignment}, \rref{riskCollaboration}), thorough review of the procedures and delivered work must be carried. The project leader should talk frequently with every member of our own team and with representatives of the subcontracted companies in order to detect possible issues and be able to act before real problems appear.
