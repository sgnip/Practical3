% -*- root: ../RiskAnalysis.tex -*-

In this chapter, we will now expose the necessary measures that should be taken to prevent the materialization of priority risks (defined in section \ref{secRiskAssesment}). We also include mitigation plans to reduce the consequences of the risks.

\section{Mitigation and prevention plans}
\label{secMitigation}

\subsection{Management plan for \rref{riskPrec}}

Taking into account that this risk exists because of our team's lack of experience in the type of application and the technical architecture, any solution involves bringing in external people. Freelance consultants should be considered at the beginning of the project to give some guidelines, and if major problems arise to solve them quickly without incurring in further delays. Of course, the time saved in any case should compensate the extra costs derived from the consultant.

Another measure that should be taken is requiring our staff to document themselves thoroughly on the architecture and application type, in order to avoid simple problems that can cause delays.

\subsection{Management plan for \rref{riskAssignment}}

A strict assignment of tasks must be enforced during the project. Communication channels must be established prior to the beginning of the project in order to avoid communication issues.

\subsection{Management plan for \rref{riskQuality}}

Given the fact that the quality can suffer from the constraints of the project, the testing plan should be taken seriously and rigorously. The project should be developed with quality standards in mind, testing as early and as frequently as possible to reduce the number of errors introduced and the mean time to their fix.

\subsection{Management plan for \rref{riskCollaboration}}

Strict conditions should be transmitted to the subcontracted companies. They must know the constraints of the project and its situation, and they must be able to deliver robust, tested and documented systems. Thorough revision of their delivered work should be done in order to avoid problems.

If the working methods of any subcontracted company don't fit in with Triforce's work, it should be communicated as soon as possible to solve it and avoid further problems. If those can't be solved, collaboration should be terminated and alternatives must be considered, such as deriving more work to the other subcontracted company or bringing in more programmers from Triforce making use of the indemnification that the non-compliant subcontracted company must pay.

\subsection{Management plan for \rref{riskDelays}}

A realistic calendar must be developed. Our staff must adhere to it, and report any significant delay. Those delays should be accommodated in the schedule of the project to avoid delays on the final delivery date.
