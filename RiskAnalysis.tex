\documentclass[11pt]{article}

\usepackage{fancyhdr} % Cabeceras de página
\usepackage{lastpage} % Módulo para añadir una referencia a la última página
\usepackage{titling} % No tengo claro para qué es esto
\usepackage[left=3cm,right=2.5cm,top=3cm,bottom=2cm]{geometry} % Márgenes
\usepackage[T1]{fontenc}
\usepackage[utf8x]{inputenc}
\usepackage{xspace}
\usepackage{graphicx}
\usepackage{tikz}
\usepackage{wrapfig}
\usepackage{hyperref}
\usepackage{amssymb}
\usepackage{multirow}
\usepackage[official]{eurosym}
\usepackage{enumitem}
\usepackage{pdfpages}
\usepackage{ifthen}
\usepackage{etoolbox} % Comandos guays.

\hypersetup{
  	hyperindex,
    colorlinks,
    allcolors=blue!60!black
}


\setcounter{secnumdepth}{2}
\renewcommand{\baselinestretch}{1.4}

\title{UAM Software Notification and Damage Management System \\ Project Plan}
\date{\today}
\author{{\Large Triforce} \\ \vspace{5pt} \textit{Iván Márquez Pardo, Víctor de Juan Sanz, Guillermo Julián Moreno}}

\fancyhf{}
\fancypagestyle{plain}{%
	\lhead{\raisebox{12pt}{\textsc{Risk Analysis} - \small Ref. TFC-UAM-01}}
	\chead{\centering \vspace{-15pt} \includegraphics[width =40 pt]{Logo.jpg}}
	\rhead{\raisebox{12pt}{\small Ver. 1.0 - \today \vspace{2pt}}}
	\cfoot{\thepage\ of \pageref{LastPage}}
	\rfoot{}
}

\newcounter{risks}[section]
\newcommand{\header}[1]{\\ \indent \textbf{#1}\hspace{10pt}}

\newcommand{\reqvertsep}{\vspace{-12pt}}

\newcommand{\riskcat}{\subparagraph{Category}}
\newcommand{\riskdesc}{\reqvertsep\subparagraph{Description}}
\newcommand{\riskprob}[1]{\reqvertsep\subparagraph{Probability} $#1 \%$}
\newcommand{\riskimpact}{\reqvertsep\subparagraph{Impact}}

\newenvironment{risk}[2][]{
	\ifthenelse{\equal{#1}{}}{}{\label{#1}}
	\refstepcounter{risks}
	\subsection{Risk \arabic{risks} - #2}
}{}

\makeindex

\begin{document}
\maketitle
\tableofcontents
\newpage
\section{Introduction}

\section{References}

\section{Project definition}

\section{Estimation, Planning and important requirements}

\section{Risk analysis}

\begin{risk}{Change of the requirements}

\riskcat Requirements

\riskdesc As in every software project, requirements may be subject to change if the client identifies new needs or clarifies previously requested features.

\riskprob{30}.

\riskimpact Very low - 1. Given that the requirements have been already approved by the client in the requirement analysis document.
\end{risk}

\begin{risk}{Localization inside buildings}
\riskcat Design

\riskdesc The design of a system that can locate an user is a difficult task. In the design stage, we can find that this is actually an impossible tasks even when our preliminary analysis revealed that it cloud be done.

\riskprob{60}

\riskimpact High - 3.5. If the design can't be done, the requirements and design will need to be redone and there will be penalizations from the client.
\end{risk}

\begin{risk}{Changes in the interface with the mobile phones}
\riskcat Design

\riskdesc One of your target system are mobile phone operating systems. These are constantly changing and one of these changes may affect our project, either because of changed APIs or new or deprecated features.

\riskprob{70}

\riskimpact Low - 1. Even when it's probable that there's an update during the development of the project, it will probably consist of minor changes and won't break compatibility.
\end{risk}

\begin{risk}{Error-prone UAM's auth server interface}
\riskcat Design

\riskdesc We don't completely trust that the interface with the UAM authentication server is well defined and correct. A defensive programming approach may be necessary.

\riskprob{30}

\riskimpact Low - 2. If this risk materializes we will need to change our programming interfaces and extend the testing phase.
\end{risk}

\begin{risk}{Unknown required performance}
\riskcat Design

\riskdesc We haven't estimated the number of users that will access the system at once. This means that there's a possibility that our server becomes overwhelmed with requests and can't attend them all.

\riskprob{10} There's a low possibility that this kind of system becomes overloaded enough to require changes in the architecture or hardware.

\riskimpact Low - 2. If the system can't accommodate all the requests, a hardware upgrade will probably be enough.
\end{risk}

\section{Risk management}
\subsection{Risk 1 - Title}
Definition of actions to do to minimize the probability of occurrence and consequences.

\section{Risk monitoring}

\section{Conclusions}

\appendix

\section{Riskology tool}

\end{document}
