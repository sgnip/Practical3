\documentclass{report}

\usepackage{fancyhdr} % Cabeceras de página
\usepackage{lastpage} % Módulo para añadir una referencia a la última página
\usepackage{titling} % No tengo claro para qué es esto
\usepackage[left=3cm,right=2.5cm,top=3cm,bottom=2cm]{geometry} % Márgenes
\usepackage[T1]{fontenc}
\usepackage[utf8x]{inputenc}
\usepackage{xspace}
\usepackage{graphicx}
\usepackage{tikz}
\usepackage{wrapfig}
\usepackage{hyperref}

\hypersetup{
  	hyperindex,
    colorlinks,
    allcolors=blue!60!black
}

\newcommand{\header}[1]{\\ \indent \textbf{#1}\hspace{10pt}}

\setcounter{secnumdepth}{3}

\title{SOFTWARE ENGINEERING PROJECT\\Assignment 1 - Reflective Document}
\date{\today}
\author{Practicals Group 2451 - Team n. 3 \\ \vspace{5pt} \textit{Iván Márquez Pardo, Víctor de Juan Sanz, Guillermo Julián Moreno}}

\fancyhf{}
\fancypagestyle{plain}{%
	\lhead{\small \itshape Assignment 1 - Reflective Document \, -\, \thedate\, -\, SEPRO}
	\rhead{\vspace{-20pt} \includegraphics[width =40 pt]{../Logo.jpg}}
	\cfoot{\thepage\ of \pageref{LastPage}}
	\rfoot{}
}

\begin{document}
\maketitle
\pagestyle{plain}
\chapter{Practical Assignment 1}

\section{Deliverable details}

\noindent
\header{Date} \today.
\header{Deliverable} Practical assignment 1.
\header{Submission date} February 23, 2015.

\section{Content and procedure}

The assignment required the development of a proposal for a maintenance management system targeted to the UAM maintainers. The \textbf{procedure} followed began with a brainstorming that produced the initial ideas for the proposal. Then we carried out a research on the Internet looking for products with similar functionalities to those we wanted to offer, and extracted new ideas or improvements to include in our own project. We divided the work between all the team members and started developing the requirements, project definition and design for the Technical Report.

The interview with a member of the maintenance system helped to better comprehend the problem and clarify some obscure details that we still had. All the information and ideas for the project we had were summarized in our presentation and written down in our Software Requirements Specification document.

As we had to deliver an oral presentation of our proposal, we prepared the slides based on what we had in our technical report. This was really advantageous and a \textbf{strength} of our procedure as we had everything prepared for the presentation, so we didn't have problems in explaining what we had done.

\section{Identified problems}

The first days were quite \textbf{chaotic} in terms of coordination (this project has been the first time the three of us have worked together in an assignment, and writing an SRS is not something we are used to do). But from the second week, we started working as a real team. This implies that for future assignments, we will be able to do our best from the beginning.

Sometimes we had some very shorts meetings or asked questions through instant messaging apps in order to clarify some parts of our assigned work and be able to continue working on them instead of saving those questions for the announced \textbf{Meeting Minutes}. The result was that not all our meetings appear on Appendix C (Meetings); only the most important and longest appear.

As a possible \textbf{improvement} of our procedure, we think that we should plan ahead and distribute the work in the first day of the assignment to avoid the \textbf{problem} of accumulated work near the deadlines.

\section{Acquired knowledge}

With this assignment we have learned to do a real analysis of a real-world problem and design a software product that potentially solves it. We never had the chance of thinking in a big application like this one and it is quite different to the software we are used to develop.

Thinking in all possible details of the application has been a good experience. We know we have missed some of them, but during the time we have been working, we realized we had missed others that ended up included in the proposal. We can imagine now with a better perspective how a Software Engineering's work is, at least regarding the Analysis and Design parts of the life cycle of a software product.

\section{Reflections on future assignments}

For our future assignments, we will have meetings after math class as we have had sometimes in this assignment: we're free during that, we all are in the same place and, even more important, those meetings were more productive than the ones had in EPS or even in class.

For future assignments of the course, we recommend having some time in all the classes to work in the assignments or at least to distribute work for the week. We would like more this than some classes with no working time, and some others were all the time is for working in our assignments.

\section{Observations}

Our project may not satisfy the minimum amount of pages stipulated in the statement, but we consider that we have written down every issue we needed in order to completely describe our project proposal.


\chapter{Practical Assignment 2}
\section{Deliverable details}

\noindent
\header{Date} \today.
\header{Deliverable} Practical assignment 2.
\header{Submission date} April 7, 2015.

\section{Content and procedure}

The assignment required the development of a project plan. The project that had to be planned was a maintenance management system targeted to the UAM maintainers.

The \textbf{procedure} began with the compilation of the requirements present in the statement and in our proposal in assignment 1. Once all the requirements were written up, we formatted them in the input-steps-output format.

When all the requirements were expressed in those terms, we discussed their complexity and applied the function points method in order to get an estimate for the size of the system. Based on that data, we applied Cocomo II using the CoStart tool to one of the subsystems to estimate its cost. We also planned the schedule using Microsoft Project, generating a Gantt diagram with the planning for the development.

The slides prepared for the presentation were based in what we had already written in the Project Plan, with lots of \LaTeX\ code being shared between the those documents. This was really advantageous and a \textbf{strength} of our procedure as we had everything prepared for the presentation, so we didn't have problems writing down and explaining what we had done.

\section{Identified problems}

One of the biggest problems we've found is the underestimation of the complexity of some parts of the assignment. Our first reading of the statement of the assignment wasn't too deep and we didn't take into account all the work that had to be done with the requirements and its adaptation to the function points method.

Because of this, we delayed the calculation of the function points and the development of the Gantt diagram. We should organize better learning how to use a planning tool under pressure is not the best thing.

As a possible \textbf{improvement} of our procedure, we think that we should understand completely the assignment before starting to work and planning ahead instead of rereading the statement and discovering aspects we forgot.

Another aspect is the tooling proposed to do Cocomo II estimations is a little bit outdated and too cryptic: there are a lot of abbreviations and technical terms that can't be understood unless one has the Cocomo II specification in another window.

\section{Acquired knowledge}

With this assignment we have learned how to plan and schedule a project given a set of requirements, using established tools and algorithms. We've made an idea of how much can a project cost and last (even though we consider our estimations a little bit overblown in terms of cost and time).

We've also learned different methods to estimate the size and cost of a project, and, what we consider the most important lesson, we've learned how to convert a set of requirements to estimations and schematics of the final system.

However, we can't help but think about the usefulness of these lessons. As we said before, the estimations seem too high. Almost 300 days of development for a project that is not excessively complex (we're talking about a CRUD\footnote{Create, Read, Update, Delete.} web application, basically a database interface) and is neither too big doesn't seem like an adjusted estimation, at least taking into account the current web technologies.

Also, the way the statement has been exposed, an iterative waterfall lifecycle model has been imposed, and this creates situations that we deem unreal. For example, we think that having coding and unit testing as separated tasks is a bad practice, or directly a show-stopper if practices like TDD\footnote{Test-Driven Development} are going to be implemented. Also, given that this is a web application and that currently the amount of continuous integration solutions for these kind of system is more than enough, planning for 4-8 days of deployment seems completely redundant if not a waste of time.

In other words, we think that, even if this assignment has had some valuable lessons, it doesn't reflect the possibilities that modern tools and practices put within our reach to speed up development while maintaining or even improving its quality.

\section{Reflections on future assignments}

For future assignments of the course, we will organize ourselves better taking into account the full statement, without underestimating any part of it.
\chapter{Practical Assignment 3}
\section{Deliverable details}

\noindent
\header{Date} \today.
\header{Deliverable} Practical assignment 3.
\header{Submission date} April 28, 2015.

\section{Content and procedure}

The assignment required the performance of a risks management plan for the project that we specified in previous practical assignments.

We began the task reviewing the questions presented in the \textit{Taxonomy-based Risk Identification} document and extracting risks based on them. Once all the risks were identified, we estimated their probability and impact and devised plans to prevent and mitigate the most important of those.

Finally, we assigned responsibilities to project participants and determined the procedure for risk monitoring and control, using Riskology to calculate delays and cancellation probabilities.

\section{Identified problems}

The biggest problem we've found is the lack of concretion in the assignment statement. For example, we didn't know how plausible should a risk be for it to be added to our document. We also didn't know what happened if our budget or time was exceeded: should our company assume the losses or cancel the project without hesitations?

We also found problems with Riskology: it doesn't work correctly in all Excel versions or other suites, such as LibreOffice, and it doesn't offer detailed information. We expected to see clearly mean delay, cancellation probability and confidence intervals for those measures, in order to assess their precision.

These questions delayed our work, as we had to wait until the classroom sessions to solve them with the teacher.

\section{Acquired knowledge}

This assignment has taught us what kind of ``disasters'' can happen in a software project, and how could we mitigate and prevent them.

However, as with previous assignments, we think that the knowledge acquired is not proportional to the time we've been working on this. Most of the complexity of the assignment has been related to writing all the necessary documentation and thinking about details like which risks should be worthy of consideration in the assignment, or the formatting of the document. We haven't spent much time thinking about impact or probability of risk realization, which we consider is the actual knowledge that we should extract from this assignment.

\section{Reflections on future assignments}

We think that the statement should specify the important details that, in this case, have been left out.

Regarding our work, task division and assignment is an aspect of our organization that we should improve, as the lack of it has caused some delays in this assignment.


\chapter{Final Reflection Document}

\section{Deliverable details}

\noindent
\header{Date} \today.
\header{Deliverable} Practical assignment 1.
\header{Submission date} February 23, 2015.

\section{Content and procedure}
\subsection{Practical assignment procedure}
We did the three practical assignments of SEPRO corresponding to the technical report, the project plan and the risk analysis.

The most time of the work have been done at home, out of the class, mainly in the first and second assignment. As the third one was shorter and we had more time in class we didn't have to work too much out of the class.

\subsection{Strengths of the practical assignment procedure}

We have been using LaTeX to generate the PDF derivable and we have used git to control versions of the documents and to allow parallel work and simultaneous modifications.

To communicate ourselves we used Telegram. It allowed us to be in constant communication (because of the mobile app) and it also allow us very efficient communication when we were working in a computer (because of the web app).

\subsection{Possible improvements of the practical assignment procedure}



\section{Comments on the evolution and progression of the practical assignments}
It has been nice to do the documentation of a project almost since the beginning. First analyzing, then planning and of course, estimating the risks. We appreciate that the last assignment is easier and shorter because we have other works to deliver at the end of the course. 

\section{Problems identified during the term (practical assignments)}
We had problems sometimes when we thought we could spend the time in the laboratory to work in some practical or optional assignment but when we got class we had some theory to see. 

We would have liked to know in advance the planning of the course, in terms of theory hours and working hours.

\section{Learning / knowledge acquired during the term (practical assignments)}

We have improved our formal written English and we have learned to use some tools (such us COCOMO or Riskology). We are not so sure that we are going to use those tools and methodologies in the future because of the incompleteness of some aspects.

\section{Reflection on and analysis of the work completed during the term}
As we mentioned before, we are not very sure that we are going to use the tools and acknowledgments learned during this term because we had some problems to know things for sure so we had to invent too much to consider useful enough this tools. 

On the other hand, we know in real projects may be useful when you dedicate your whole journey to the same project and you are going to spend the next \emph{x} time of your work, so you better estimate that \emph{x} good enough. 

\section{Justification of the major and hardest portfolio contents}
We haven't used (at least conscientiously) the portfolio technique.

\section{Skills developed as a result of the work performed throughout the term}


\section{Importance and applicability of portfolio production for your future profession}
We haven't used (at least conscientiously) the portfolio technique.

\section{Observations}
We think it would be much more interesting and quite more didactic if all this analysis and planning would be used. We think if after the analysis we would have coded the project, we could have learned if that estimation was good. Spanish wisdom says ``De los errores se aprende'', that means ``from errors we learn''. If we just make a planning that we are not going to follow, we can't make any mistake and we miss all that learning we can get from the mistakes we can make.

\end{document}
