\section{Deliverable details}

\noindent
\header{Date} \today.
\header{Deliverable} Practical assignment 2.
\header{Submission date} April 7, 2015.

\section{Content and procedure}

The assignment required the development of a project plan. The project that had to be planned was a maintenance management system targeted to the UAM maintainers.

The \textbf{procedure} began with the compilation of the requirements present in the statement and in our proposal in assignment 1. Once all the requirements were written up, we formatted them in the input-steps-output format.

When all the requirements were expressed in those terms, we discussed their complexity and applied the function points method in order to get an estimate for the size of the system. Based on that data, we applied Cocomo II using the CoStart tool to one of the subsystems to estimate its cost. We also planned the schedule using Microsoft Project, generating a Gantt diagram with the planning for the development.

The slides prepared for the presentation were based in what we had already written in the Project Plan, with lots of \LaTeX\ code being shared between the those documents. This was really advantageous and a \textbf{strength} of our procedure as we had everything prepared for the presentation, so we didn't have problems writing down and explaining what we had done.

\section{Identified problems}

One of the biggest problems we've found is the underestimation of the complexity of some parts of the assignment. Our first reading of the statement of the assignment wasn't too deep and we didn't take into account all the work that had to be done with the requirements and its adaptation to the function points method.

Because of this, we delayed the calculation of the function points and the development of the Gantt diagram. We should organize better learning how to use a planning tool under pressure is not the best thing.

As a possible \textbf{improvement} of our procedure, we think that we should understand completely the assignment before starting to work and planning ahead instead of rereading the statement and discovering aspects we forgot.

Another aspect is the tooling proposed to do Cocomo II estimations is a little bit outdated and too cryptic: there are a lot of abbreviations and technical terms that can't be understood unless one has the Cocomo II specification in another window.

\section{Acquired knowledge}

With this assignment we have learned how to plan and schedule a project given a set of requirements, using established tools and algorithms. We've made an idea of how much can a project cost and last (even though we consider our estimations a little bit overblown in terms of cost and time).

We've also learned different methods to estimate the size and cost of a project, and, what we consider the most important lesson, we've learned how to convert a set of requirements to estimations and schematics of the final system.

However, we can't help but think about the usefulness of these lessons. As we said before, the estimations seem too high. Almost 300 days of development for a project that is not excessively complex (we're talking about a CRUD\footnote{Create, Read, Update, Delete.} web application, basically a database interface) and is neither too big doesn't seem like an adjusted estimation, at least taking into account the current web technologies.

Also, the way the statement has been exposed, an iterative waterfall lifecycle model has been imposed, and this creates situations that we deem unreal. For example, we think that having coding and unit testing as separated tasks is a bad practice, or directly a show-stopper if practices like TDD\footnote{Test-Driven Development} are going to be implemented. Also, given that this is a web application and that currently the amount of continuous integration solutions for these kind of system is more than enough, planning for 4-8 days of deployment seems completely redundant if not a waste of time.

In other words, we think that, even if this assignment has had some valuable lessons, it doesn't reflect the possibilities that modern tools and practices put within our reach to speed up development while maintaining or even improving its quality.

\section{Reflections on future assignments}

For future assignments of the course, we will organize ourselves better taking into account the full statement, without underestimating any part of it.