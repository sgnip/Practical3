\section{Deliverable details}

\noindent
\header{Date} \today.
\header{Deliverable} Practical assignment 3.
\header{Submission date} April 28, 2015.

\section{Content and procedure}

The assignment required the performance of a risks management plan for the project that we specified in previous practical assignments.

We began the task reviewing the questions presented in the \textit{Taxonomy-based Risk Identification} document and extracting risks based on them. Once all the risks were identified, we estimated their probability and impact and devised plans to prevent and mitigate the most important of those.

Finally, we assigned responsibilities to project participants and determined the procedure for risk monitoring and control, using Riskology to calculate delays and cancellation probabilities.

\section{Identified problems}

The biggest problem we've found is the lack of concretion in the assignment statement. For example, we didn't know how plausible should a risk be for it to be added to our document. We also didn't know what happened if our budget or time was exceeded: should our company assume the losses or cancel the project without hesitations?

We also found problems with Riskology: it doesn't work correctly in all Excel versions or other suites, such as LibreOffice, and it doesn't offer detailed information. We expected to see clearly mean delay, cancellation probability and confidence intervals for those measures, in order to assess their precision.

These questions delayed our work, as we had to wait until the classroom sessions to solve them with the teacher.

\section{Acquired knowledge}

This assignment has taught us what kind of ``disasters'' can happen in a software project, and how could we mitigate and prevent them.

However, as with previous assignments, we think that the knowledge acquired is not proportional to the time we've been working on this. Most of the complexity of the assignment has been related to writing all the necessary documentation and thinking about details like which risks should be worthy of consideration in the assignment, or the formatting of the document. We haven't spent much time thinking about impact or probability of risk realization, which we consider is the actual knowledge that we should extract from this assignment.

\section{Reflections on future assignments}

We think that the statement should specify the important details that, in this case, have been left out.

Regarding our work, task division and assignment is an aspect of our organization that we should improve, as the lack of it has caused some delays in this assignment.