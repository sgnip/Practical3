
\section{Deliverable details}

\noindent
\header{Date} \today.
\header{Deliverable} Practical assignment 1.
\header{Submission date} February 23, 2015.

\section{Content and procedure}

The assignment required the development of a proposal for a maintenance management system targeted to the UAM maintainers. The \textbf{procedure} followed began with a brainstorming that produced the initial ideas for the proposal. Then we carried out a research on the Internet looking for products with similar functionalities to those we wanted to offer, and extracted new ideas or improvements to include in our own project. We divided the work between all the team members and started developing the requirements, project definition and design for the Technical Report.

The interview with a member of the maintenance system helped to better comprehend the problem and clarify some obscure details that we still had. All the information and ideas for the project we had were summarized in our presentation and written down in our Software Requirements Specification document.

As we had to deliver an oral presentation of our proposal, we prepared the slides based on what we had in our technical report. This was really advantageous and a \textbf{strength} of our procedure as we had everything prepared for the presentation, so we didn't have problems in explaining what we had done.

\section{Identified problems}

The first days were quite \textbf{chaotic} in terms of coordination (this project has been the first time the three of us have worked together in an assignment, and writing an SRS is not something we are used to do). But from the second week, we started working as a real team. This implies that for future assignments, we will be able to do our best from the beginning.

Sometimes we had some very shorts meetings or asked questions through instant messaging apps in order to clarify some parts of our assigned work and be able to continue working on them instead of saving those questions for the announced \textbf{Meeting Minutes}. The result was that not all our meetings appear on Appendix C (Meetings); only the most important and longest appear.

As a possible \textbf{improvement} of our procedure, we think that we should plan ahead and distribute the work in the first day of the assignment to avoid the \textbf{problem} of accumulated work near the deadlines.

\section{Acquired knowledge}

With this assignment we have learned to do a real analysis of a real-world problem and design a software product that potentially solves it. We never had the chance of thinking in a big application like this one and it is quite different to the software we are used to develop.

Thinking in all possible details of the application has been a good experience. We know we have missed some of them, but during the time we have been working, we realized we had missed others that ended up included in the proposal. We can imagine now with a better perspective how a Software Engineering's work is, at least regarding the Analysis and Design parts of the life cycle of a software product.

\section{Reflections on future assignments}

For our future assignments, we will have meetings after math class as we have had sometimes in this assignment: we're free during that, we all are in the same place and, even more important, those meetings were more productive than the ones had in EPS or even in class.

For future assignments of the course, we recommend having some time in all the classes to work in the assignments or at least to distribute work for the week. We would like more this than some classes with no working time, and some others were all the time is for working in our assignments.

\section{Observations}

Our project may not satisfy the minimum amount of pages stipulated in the statement, but we consider that we have written down every issue we needed in order to completely describe our project proposal.

